\chapter*{Glossaire}
\addcontentsline{toc}{chapter}{Glossaire}

\begin{description}
\item[API Win32 :] Ancêtre du framework .NET.
\item[Base de tests :] Ensemble d'images traitées par l'application de tests pour mettre à l'épreuve GdPicture SDK.
\item[C# :] Langage principal du framework .NET. C'est un langage de haut niveau très semblable à Java.
\item[Fichier témoin ] Associé à une image de la base de tests, il contient les valeurs des codes-barres qui y sont contenus. Sert à contrôler la valeur détectée par GdPicture.
\item[Framework .NET :] Ensemble d'outils et de fonctions sur lesquels est basé le système d'exploitation Windows et permettant de créer des applications intégrées à ce système.
\item[Hook :] Mécanisme bas niveau d'interception des messages systèmes comme les événements clavier et souris.
\item[JSON :] Langage de mise en forme des données à la syntaxe très simple. Permet de stocker des données dans un fichier aussi bien lisible par un humain que par un ordinateur.
\item[Rapport :] Fichier JSON contenant les informations d'une détection entière de la base de tests par l'application de tests.
\item[SDK :] Ensemble d'outils pour développeurs répondant à un besoin particulier.
\item[Thread :] En français \og Fil d'exécution \fg{}. Séparer certains calculs sur plusieurs threads permet au système d'exploitation de les effectuer en même temps lorsque l'ordinateur est équipé d'un processeur le permettant.
\item[Visual Studio :] Environnement de développement pour les applications Windows. Contient un éditeur de texte avancé, un compilateur et des outils de dessins d'interfaces graphiques.
\item[WPF :] Spécification graphique du framework .NET permettant d'unifier l'aspect des fenêtres dans le système et les applications Windows.
\item[XAML :] Langage dérivé de XML permettant la mise en forme des fenêtres WPF.
\end{description}

\chapter*{Médiagraphie}
\addcontentsline{toc}{chapter}{Médiagraphie}

\begin{description}
\item[Réseau des développeurs Microsoft :] http://msdn.microsoft.com/en-US/, véritable mine d'informations sur les outils Microsoft.
\item[Site du zéro :] http://www.siteduzero.com/, aux forums très actifs.
\item[CodeProject :] http://www.codeproject.com/, plusieurs tutoriels intéressants et des projets d'exemple aux sources ouvertes.
\item[CodePlex :] http://www.codeplex.com/, également grosse banque de projets OpenSource.

\end{description}