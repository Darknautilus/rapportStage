\chapter*{Résumé}
\addcontentsline{toc}{chapter}{Résumé}

Afin de clore mes deux années d'études d'Informatique à l'IUT de Blagnac, j'ai effectué un stage dans une petite entreprise nommée ORPALIS. Ce stage a duré du 8 avril au 14 juin 2013. Le but de ce stage était de réaliser une application de tests de non-régression pour un moteur de détection de codes-barres. Une telle application serait utilisée par l'équipe d'ORPALIS pour avoir des statistiques sur les performances du moteur et d'analyser leur évolution tout au long des nouvelles versions du logiciel. L'environnement de travail était un environnement Microsoft, j'ai donc utilisé le logiciel Visual Studio 2012 ainsi que le langage C\# pour accomplir ma tâche. La compréhension du framework .NET et de l'écosystème centré autour du système d'exploitation Windows m'a permis de créer une application à la fois puissante et simple d'utilisation.

Une fois ce premier projet terminé, j'ai complété mes connaissances en créant une deuxième application tournée cette fois vers le grand public servant à lire des codes-barres. Cette application est conçue pour rendre la détection de codes-barres aussi simple qu'avec un lecteur filaire.

Ces deux projets ont été l'occasion de découvrir le travail au sein d'une petite équipe et m'ont permis d'affiner mon projet professionnel. Je suis prêt à entreprendre des études en école d'ingénieurs afin de développer une expertise dans le génie logiciel.