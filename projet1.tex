\chapter{Une application de test à grande envergure}

Mon stage débute le 8 avril. Une fois arrivé dans les locaux d'Orpalis à Colomiers, Loïc et Cédric me présentent l'entreprise et ses objectifs.
C'est en comprenant l'envergure de GdPicture que je comprends la nécessité d'automatiser certains tests afin d'obtenir une visibilité en terme de fiabilité du SDK.

\section{Objectifs}

Loïc me présente une base de test contenant près d'un millier d'images de codes-barres classés par type.
Mon objectif principal est d'écrire une application qui, en utilisant GdPicture, va passer en revue chaque image et tenter d'y détecter un code-barres.
Il s'agira d'enregistrer les résultats des détections, afin d'émettre des rapports sur la quantité d'images lues et le nombre de codes-barres détectés.
L'idée est d'effectuer une analyse complète de la base de test à chaque nouvelle version du moteur de détection du SDK, afin de comparer les résultats ; il faudra donc envisager un mécanisme de comparaison des différents rapports entre eux.

Les tests sont une activité qui prend du temps et des ressources, c'est pour cela qu'une machine puissante tourne en permanence aux bureaux d'Orpalis.
Cette machine contient un processeur capable d'effectuer plusieurs tâches en parallèle. Chaque tâche parallèle est exécutée sur ce que l'on appelle un fil d'exécution (en anglais, \og thread\fg ).
Ainsi, pour exploiter pleinement les performances d'une telle machine, il est nécessaire de concevoir son application pour qu'elle utilise plusieurs threads. On dit de cette application qu'elle est multi-threadée.

Il est bien entendu que la base de test dont je dispose n'est qu'un fragment de la vraie base de test qui contient, elle, bien plus d'images.
Il est impensable de ne traiter les images qu'une à une, c'est pourquoi Loïc me fournit un modèle d'application multi-threadée sur lequel je base mon développement.

\section{Développement}



\section{Bilan}