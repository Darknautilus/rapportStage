\chapter{Un lecteur virtuel de codes-barres}

Après que l'application de test ait été achevée, je me retrouvai sans activité alors qu'il me restait la moitié de mon stage.
C'est ainsi que je me lançai dans un projet de détecteur virtuel de codes-barres.
Me basant sur le moteur de détection de GdPicture que je connaissais désormais, j'ai développé une application ergonomique et simple qui simulerai un lecteur de codes-barres sur des images ou des documents scannés.

\section{Objectifs}

Cette application devait tout d'abord reproduire le fonctionnement des lecteurs de codes-barres USB existants.
Il devait être possible d'accéder rapidement à l'application depuis n'importe quelle autre, et que le lecteur soit une extension de la souris.
Enfin, plusieurs actions devaient être possibles une fois un code-barres détecté.

\section{Développement}

J'ai choisi de créer une application sans fenêtre, c'est à dire qui s'exécuterait d'une manière transparente pour l'utilisateur, et accessible depuis la zone de notifications.
J'ai pour cela utilisé une bibliothèque WPF\footnote{Disponible ici : http://www.hardcodet.net/projects/wpf-notifyicon} prenant en charge cette fonctionnalité.


\section{Bilan}