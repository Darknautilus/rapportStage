\chapter{ORPALIS}

\section{Présentation}

ORPALIS est fondée en 2003 par Loïc Carrère, alors âgé de vingt-deux ans et encore employé à Sodifrance\footnote{Sodifrance est une SSII (Société de Services en Ingénierie Informatique) dont le siège social se situe près de Rennes.}. Son objectif est de donner un cadre professionnel à son projet \emph{GdPicture} qu'il développe sur son temps libre. La forme juridique de l'entreprise est une SARL\footnote{Société A Responsabilité Limitée}. 

Après plus de neuf ans d'activité, la santé d'ORPALIS est au mieux, profitant comme beaucoup d'autres start-ups, d'une croissance à deux chiffres. \emph{GdPicture}, dont l'équipe prépare actuellement la dixième version, a été largement adopté dans le monde professionnel puisqu'aujourd'hui, plus de douze mille développeurs dans soixante-dix pays l'utilisent. Plusieurs autres produits basés sur \emph{GdPicture} ont été édités, assurant à l'entreprise une forte présence dans le domaine du traitement d'images.

Le 10 mai 2013, ORPALIS annonce l'obtention du statut de Jeune Entreprise Innovante décerné par le Ministère de la Recherche. Ce statut promet une augmentation de l'activité de recherche de l'entreprise en terme de traitement d'image et de gestion de documents.

\section{Équipe}

Orpalis est constituée de quatre personnes dans les bureaux de Colomiers :

\begin{description}
\item[Loïc Carrère] Co-gérant et responsable du développement. Il est seul à écrire du code pour les différents projets d'ORPALIS. 
\item[Cédric Grard] Responsable support. Il s'occupe du signalement de bugs et des questions des clients sur le plan technique.
\item[\'{E}lodie Tellier] Co-gérante, responsable des ventes et chargée de communication. Elle assiste Cédric dans le support client, mais sur les questions de licence et de ventes.
\item[Caroline Tellier] Elle assiste \'{E}lodie dans le support commercial, sur le forum et le chat.
\end{description}

L'entreprise emploie également d'autres personnes dans plusieurs pays :

\begin{itemize}
\item Une dans le nord de la France
\item Deux en Roumanie
\item Une en Jordanie
\end{itemize}

\section{Produits}

\subsection{GdPicture}

GdPicture est le fer de lance d'ORPALIS. Il s'agit d'un SDK\footnote{Software Development Kit, plus de détails en page \pageref{gdpicture}} à destination des développeurs d'applications finales.

\subsection{PaperScan}

Lancé en 2011, PaperScan est un logiciel de numérisation prenant en charge un grand nombre de scanners. Il permet également de grandes possibilités de retouche d'images grâce à GdPicture, ainsi que l'export dans un grand nombre de formats.

\subsection{PDF Reducer}

En décembre 2012, ORPALIS lance son nouveau produit à destination des utilisateurs finaux : PDF Reducer. Ce logiciel affiche des grandes performances en terme de compression. Il concurrence sérieusement d'autres produits similaires comme Adobe Reader.