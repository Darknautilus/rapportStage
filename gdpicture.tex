\chapter{GdPicture}
\label{gdpicture}

\section{Qu'est-ce qu'un SDK ?}

Un kit de développement logiciel (en anglais : Software Development Kit) est un ensemble de bibliothèques logicielles permettant de faire abstraction d'une grande partie d'opérations complexes. Une bibliothèque se constitue de plusieurs fonctions répondant à un besoin particulier.

GdPicture SDK fournit un grand nombre d'outils pour développeurs dans le domaine du traitement d'images.

\section{Fonctionnalités principales}

\begin{itemize}
\item Gestion complète du format PDF (affichage, annotations, éditions des métadonnées)
\item Gestion d'un grand nombre (90) de types d'images
\item Controle de scanners
\item Création d'images animées (GIF)
\item Reconnaissance optique de caractères (OCR)
\item Lecture et écriture de codes-barres (1D, Datamatrix, PDF417 et QR Code)
\end{itemize}

Cette liste n'est pas exhaustive, et toutes les fonctionnalités ne sont pas disponibles suivant la version que l'on possède. En effet, il est possible d'acquérir le c\oe ur de GdPicture pour ensuite lui ajouter des plugins correspondant à ses besoins. On trouvera  par exemple un plugin par type de code-barres, un plugin de numérisation, etc.

\section{Utilisation}

Afin d'utiliser GdPicture, il faut d'abord enregistrer une clé de licence. Il s'agit d'une mesure de protection empêchant ceux qui n'auraient pas acheté le SDK de l'utiliser. L'enregistrement se fait avec cette instruction :

\verb|new GdPicture9.LicenseManager().lm.RegisterKEY("ma_cle");|

Dans l'application j'utilise un objet de type \verb|GdPictureImaging| qui va me servir à appeler plusieurs fonctions de traitement d'images, dont la détection de codes-barres.